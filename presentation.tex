\documentclass[a4paper]{article}

%% Language and font encodings
\usepackage{gb4e}
\noautomath
\usepackage{natbib}
\usepackage{rotating}
\usepackage[english]{babel}
\usepackage[utf8x]{inputenc}
\usepackage[T1]{fontenc}

%% Sets page size and margins
%\usepackage[a4paper,top=3cm,bottom=2cm,left=3cm,right=3cm,marginparwidth=1.75cm]{geometry}

%% Useful packages
\usepackage{amsmath}
\usepackage{graphicx}
\usepackage[colorinlistoftodos]{todonotes}
\usepackage[colorlinks=true, allcolors=blue]{hyperref}
\usepackage{fixltx2e}
\usepackage{Sweave}

\newcommand*{\myfont}{\fontfamily{ccr}\selectfont}
%the newcommand change the selected word into 'ccr' font
%e.g. \begin{myfont} multi-bleu.perl \end{myfont}

\title{Building Neural Net Machine Translation Systems Using Interlinear Glossed Texts}
\author{Yuan-Lu Chen}


\begin{document}
\maketitle
\Sconcordance{concordance:presentation.tex:presentation.rnw:%
1 95 1}
\Sconcordance{concordance:presentation.tex:./alcComplete_table.Rnw:ofs 96:%
11 23 0}
\Sconcordance{concordance:presentation.tex:presentation.rnw:ofs 120:%
98 1 1}


\begin{exe}
\ex 
\begin{xlist}  
\ex Training Data:
	\gll    Tha a athair nas sine na a mh\`athair.\\  
            be.pres 3sm.poss father comp old.cmpr comp 3sm.poss mother
\\  
    \glt    `His father is older than his mother.' 
\ex Test Data:
\gll Tha mi a' sireadh an leabhair bhig ghuirm.\\
be-PRES-IND 1S PROG searching-VN ART book-G small-G blue-G\\
\glt `I am looking for the small blue book.' 
\end{xlist}
\end{exe}

\begin{exe} 
\ex \textbf{Gloss-helps hypothesis: the translation systems trained with the gloss data incorporated should outperform the systems trained with only Gaelic and English sentences pairs (i.e. without gloss data).}

The hypothesis can have two versions, strong and weak:
	\begin{xlist}
	\ex \label{strong_hy} Strong version: Gloss may replace the source natural language totally, and the system outperforms the system trained with source natural language to target language sentence pairs (i.e. the baseline systems).  
	\ex \label{weak_hy} Weak version: Gloss only increases the performance of the baseline systems, but cannot replace the source language.
	\end{xlist}
\end{exe}


\begin{exe}  
\ex Baseline: 
\begin{xlist}
	\ex Trained with:\\
	 Gaelic sentence $\rightarrow$ English Sentence 
	\begin{xlist}
		\ex e.g. \\
			Tha a athair nas sine na a mh\`athair. \\
			$\rightarrow$ \\
			His father is older than his mother.
	\end{xlist}
	\ex Test with:\\
		Tha mi a' sireadh an leabhair bhig ghuirm.\\
		$\rightarrow$ \\
		Predicted\_English\_Translation\textsubscript{j}
	\ex Evaluation of the model:\\
		compare\\ Predicted\_English\_Translation\textsubscript{j}\\ with\\
		`I am looking for the small blue book.'  
\end{xlist}
\end{exe}


\begin{exe}  
\ex Gloss Treatment: 
\begin{xlist}
	\ex Trained with:\\
	 Gloss line $\rightarrow$ English Sentence 
	\begin{xlist}
		\ex e.g. \\
			be.pres 3sm.poss father comp old.cmpr comp 3sm.poss mother \\
			$\rightarrow$ \\
			His father is older than his mother.
	\end{xlist}
	\ex Test with:\\
		be-PRES-IND 1S PROG searching-VN ART book-G small-G blue-G\\
		$\rightarrow$ \\
		Predicted\_English\_Translation\textsubscript{j}
\end{xlist}
\end{exe}

% !Rnw root = cake_chapter.Rnw
% latex table generated in R 3.4.4 by xtable 1.8-2 package
% Wed Apr 18 13:56:55 2018
\begin{table}[ht]
\centering
\begin{tabular}{lcc}
  \hline
Round & Gaelic (Baseline) & GLOSS \\ 
  \hline
0 & 17.29 & 18.39 \\ 
  1 & 16.42 & 18.00 \\ 
  2 & 15.29 & 16.02 \\ 
  3 & 15.97 & 20.22 \\ 
  4 & 17.79 & 19.02 \\ 
  5 & 16.73 & 15.53 \\ 
  6 & 17.11 & 18.00 \\ 
  7 & 16.37 & 20.08 \\ 
  8 & 15.93 & 15.82 \\ 
  9 & 16.99 & 15.93 \\ 
   \hline
Mean & 16.59 & 17.70 \\ 
   \hline
\end{tabular}
\caption{BLEU scores of Model\textsubscript{GDtoEN} and Model\textsubscript{GLOSStoEn}} 
\label{Table:GLOSS}
\end{table}
The average score of the Models\textsubscript{GLOSStoEN} is only slightly higher than the average score of the Models\textsubscript{GDtoEN}.
Also, after doing a paired T-test, the difference between the two types of models is not attested
(M\textsubscript{GDToEn}=16.59, SD\textsubscript{GDToEn}=0.74; M\textsubscript{GLOSStoEN}=17.70, SD\textsubscript{GLOSStoEN}=1.78; t(9)=1.97, p=0.080)

\begin{exe}  
\ex Parallel-Partial Treatment: 
\begin{xlist}
	\ex Trained with:\\
	 Gaelic sentence $\rightarrow$ English Sentence\\ 
	 Gloss line $\rightarrow$ English Sentence \\
	 Gloss line $\rightarrow$ Gaelic sentence\\
	 Gaelic word $\rightarrow$ Gloss 
	 \ex Parallel
	\begin{xlist}
		\ex Gaelic to English: \\<``Tha a athair nas sine na a mh\`athair'', ``His father is older than his mother.''>
		\ex Gloss to English: \\<``be.pres 3sm.poss father comp old.cmpr comp 3sm.poss mother'', ``His father is older than his mother''>
		\ex Gloss to Gaelic: \\<``be.pres 3sm.poss father comp old.cmpr comp 3sm.poss mother'', ``Tha a athair nas sine na a mh\`athair''>
	\end{xlist}
	\ex Partial
	\begin{xlist}
		\ex <``Tha'', ``be.pres''>
		\ex <``a'', ``3sm.poss''>
		\ex <``athair'', ``father''>
		\ex <``nas'', ``comp''>
		\ex <``sine'', ``old.cmpr''>
		\ex <``na'', ``comp''>
		\ex <``a'', ``3sm.poss''>
		\ex <``mh\`athair'', ``mother''>
	\end{xlist}
	\ex Test with:\\
		Tha mi a' sireadh an leabhair bhig ghuirm.\\
		$\rightarrow$ \\
		Predicted\_English\_Translation\textsubscript{j}
\end{xlist}
\end{exe}

% !Rnw root = cake_chapter.Rnw
% latex table generated in R 3.4.4 by xtable 1.8-2 package
% Wed Apr 18 13:56:55 2018
\begin{table}[ht]
\centering
\begin{tabular}{lcc}
  \hline
Round & Gaelic (Baseline) & ParaPart \\ 
  \hline
0 & 17.29 & 32.64 \\ 
  1 & 16.42 & 32.28 \\ 
  2 & 15.29 & 29.94 \\ 
  3 & 15.97 & 31.18 \\ 
  4 & 17.79 & 32.83 \\ 
  5 & 16.73 & 31.11 \\ 
  6 & 17.11 & 32.19 \\ 
  7 & 16.37 & 33.52 \\ 
  8 & 15.93 & 30.93 \\ 
  9 & 16.99 & 34.35 \\ 
   \hline
Mean & 16.59 & 32.10 \\ 
   \hline
\end{tabular}
\caption{BLEU scores of Model\textsubscript{GDtoEN} and Model\textsubscript{ParaParttoEn}} 
\label{Table:ParaPart}
\end{table}
M\textsubscript{GDToEn}=16.59, SD\textsubscript{GDToEn}=0.74; M\textsubscript{ParaPart}=32.10, SD\textsubscript{ParaPart}=1.33; t(9)=48.95, p<0.01.

\newpage
\begin{exe}
\ex Many other possible ways to blend Gaelic and gloss:
% latex table generated in R 3.4.4 by xtable 1.8-2 package
% Wed Apr 18 13:56:55 2018
\begin{table}[ht]
\centering
\begin{tabular}{lrrrrrrr}
  \hline
\begin{sideways} Round \end{sideways} & \begin{sideways} Baseline \end{sideways} & \begin{sideways} GLOSS \end{sideways} & \begin{sideways} ParaPart \end{sideways} & \begin{sideways} Para \end{sideways} & \begin{sideways} Interleaving \end{sideways} & \begin{sideways} Concat \end{sideways} & \begin{sideways} Google Translation \end{sideways} \\ 
  \hline
0 & 17.29 & 18.39 & 32.64 & 25.42 & 13.67 & 15.42 & 22.09 \\ 
  1 & 16.42 & 18.00 & 32.28 & 25.32 & 12.49 & 14.31 & 25.38 \\ 
  2 & 15.29 & 16.02 & 29.94 & 20.72 & 11.01 & 15.38 & 23.72 \\ 
  3 & 15.97 & 20.22 & 31.18 & 22.22 & 12.33 & 14.18 & 23.21 \\ 
  4 & 17.79 & 19.02 & 32.83 & 24.27 & 12.56 & 18.63 & 22.31 \\ 
  5 & 16.73 & 15.53 & 31.11 & 24.55 & 12.13 & 14.89 & 23.41 \\ 
  6 & 17.11 & 18.00 & 32.19 & 27.03 & 11.55 & 15.16 & 24.53 \\ 
  7 & 16.37 & 20.08 & 33.52 & 25.34 & 12.78 & 15.20 & 22.78 \\ 
  8 & 15.93 & 15.82 & 30.93 & 24.24 & 12.43 & 15.50 & 25.67 \\ 
  9 & 16.99 & 15.93 & 34.35 & 25.96 & 11.65 & 15.72 & 23.42 \\ 
   \hline
Mean & 16.59 & 17.70 & 32.10 & 24.51 & 12.26 & 15.44 & 23.65 \\ 
   \hline
\end{tabular}
\caption{BLEU scores of the treatments: Ten rounds of repeated random sub-sampling validation are conducted. For each round, the same sets of IGTs are used. Each column is a treatment, and each row is a single round of repeated random sub-sampling validation. The last column is the scores of Google Translation. We used a free Google translation API \citep{google_api} to translate the same set of test Gaelic sentences into English.} 
\label{table:complete_table}
\end{table}\end{exe}

\bibliographystyle{te}

\bibliography{ref}

\end{document}
