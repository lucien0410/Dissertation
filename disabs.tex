\begin{abstract}
Interlinear Glossed Text is widely used in linguistic studies. (1) is an example of Scottish Gaelic IGT.
\begin{exe}  
\ex \gll    Tha a athair nas sine na a mh\`athair.\\  
            be.pres 3sm.poss father comp old.cmpr comp 3sm.poss mother
\\  
    \glt    `His father is older than his mother.'  
\end{exe}

In a simple form of Interlinear Glossed Text, the first line is a sentence of the language of interest, the second line is a word-by-word translation, annotated with relevant grammatical information, and the third line is an English translation (see \citet{bickel2008leipzig} for the complete formats and options of IGT).  

The Innovation is to incorporate the gloss information of Interlinear Glossed Text data into neural net machine translation systems.
 
Critically, if the Gaelic data and the gloss data are combined in a specific way as the training data, I term which as Parallel-Partial treatment the performance of the systems is improved significantly. The Parallel-Partial treatment lets machine to learn three sets of mappings: 1.) from source sentence to target sentence, 2) from gloss lines to target sentences, and 3) source language words to glosses. 

Moreover, the boosting effect of the Parallel-Partial treatment is consistent across different languages and across neural net machine translation systems with different hyper-parameter settings. 

How theoretical linguistics may work hand in hand with natural language processing, and how neural net machine learning may exploit linguistics are important questions in both fields \citet{pater2017generative}. In addition to practically building better machine translation systems, the current work also exemplifies how theoretical linguistics may work hand in hand with natural language processing successfully.   
\end{abstract}
