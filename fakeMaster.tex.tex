\documentclass[a4paper]{article}

%% Language and font encodings
\usepackage{gb4e}
\noautomath
\usepackage{natbib}
\usepackage[english]{babel}
\usepackage[utf8x]{inputenc}
\usepackage[T1]{fontenc}

%% Sets page size and margins
%\usepackage[a4paper,top=3cm,bottom=2cm,left=3cm,right=3cm,marginparwidth=1.75cm]{geometry}

%% Useful packages
\usepackage{amsmath}
\usepackage{graphicx}
\usepackage[colorinlistoftodos]{todonotes}
\usepackage[colorlinks=true, allcolors=blue]{hyperref}
\usepackage{fixltx2e}
\usepackage{Sweave}


\title{Chapter N: Experimenting Interlinear Glossing Text}
\author{Yuan-Lu Chen}

\begin{document}
\Sconcordance{concordance:fakeMaster.tex:fakeMaster.Rnw:%
1 33 1}
\Sconcordance{concordance:fakeMaster.tex:./RealCake.Rnw:ofs 34:%
1 154 1}
\Sconcordance{concordance:fakeMaster.tex:./GLOSS_table.Rnw:ofs 189:%
1 1 17 23 0}
\Sconcordance{concordance:fakeMaster.tex:./RealCake.Rnw:ofs 214:%
157 74 1}
\Sconcordance{concordance:fakeMaster.tex:./ParaPart_table.Rnw:ofs 289:%
1 1 17 23 0}
\Sconcordance{concordance:fakeMaster.tex:./RealCake.Rnw:ofs 314:%
233 30 1}
\Sconcordance{concordance:fakeMaster.tex:./Para_table.Rnw:ofs 345:%
1 1 17 23 0}
\Sconcordance{concordance:fakeMaster.tex:./RealCake.Rnw:ofs 370:%
265}
\Sconcordance{concordance:fakeMaster.tex:./GDParaParaPart_null_table.Rnw:ofs 371:%
1}
\Sconcordance{concordance:fakeMaster.tex:./RealCake.Rnw:ofs 372:%
267 2 1}
\Sconcordance{concordance:fakeMaster.tex:./GDParaParaPart_table.Rnw:ofs 375:%
1 1 19 23 0}
\Sconcordance{concordance:fakeMaster.tex:./RealCake.Rnw:ofs 400:%
271 18 1}
\Sconcordance{concordance:fakeMaster.tex:./interleavingGdGLOSS_table.Rnw:ofs 419:%
1 1 17 23 0}
\Sconcordance{concordance:fakeMaster.tex:./RealCake.Rnw:ofs 444:%
291 21 1}
\Sconcordance{concordance:fakeMaster.tex:./concat_table.Rnw:ofs 466:%
1 1 17 23 0}
\Sconcordance{concordance:fakeMaster.tex:./RealCake.Rnw:ofs 491:%
314 16 1}
\Sconcordance{concordance:fakeMaster.tex:./ReplacingGaelic_table.Rnw:ofs 508:%
1 1 17 23 0}
\Sconcordance{concordance:fakeMaster.tex:./RealCake.Rnw:ofs 533:%
332 1 1}
\Sconcordance{concordance:fakeMaster.tex:./ReplacingGLOSS_table.Rnw:ofs 535:%
1 1 17 23 0}
\Sconcordance{concordance:fakeMaster.tex:./RealCake.Rnw:ofs 560:%
335 5 1 1 11 23 0 1 2 18 1}
\Sconcordance{concordance:fakeMaster.tex:fakeMaster.Rnw:ofs 609:%
36 5 1}



\maketitle
% !Rnw root = fakeMaster.Rnw

(

\textbf{Assuming that in the previous chapters the following points are addressed already:} 
\begin{itemize}
\item The nature of glosses has been well-explained  (Target audience: CS people without any formal linguistics background):
	\begin{itemize}
	\item What glosses are: A basic intro of interlinear gloss for non-linguists
    \item The golden nature of glosses (encodes NON-LINEAR syntax (i.e. structure parse) and semantics information) 
    \item The potential of gloss:	
		\begin{itemize}
		\item potential: providing disambiguation, labeling important grammar morphemes in the source language, providing morphological analysis, providing one-to-many and many-to-one relations of source tokens and target tokens.  
		\end{itemize}
	\end{itemize}
\item A history of machine translation, and a non-mathy description of the methods of doing machine translation. (Target reader: theoretical linguists)
\end{itemize}


)

\section{Introduction}
The Innovation is to incorporate the gloss information of Interlinear Glossed Text data into machine translation. 

In supervised machine learning models, two factors effects the performance of the trained systems:
a.) the quality of the training data and b.) the choices of the features. The properties of the gloss data as described in *CHAPTERXYZ* make it a better training data than natural language data (Scottish Gaelic in the current case) for the following reasons. 
First, glosses are more purified that natural language words. 
The most ideal meaning representation system should be built with one-meaning-to-one-representation mappings; in other words, a meaning is mapped to one and only one representation. 
Natural languages fail to do so, given that synonyms and ambiguous words/phrases are ubiquitous in natural languages. Glosses provide this one-to-one mapping. 
Second, the gloss data provides hierarchical (non-linear) syntactic parsing information to some degree. To determine what the gloss of a word is, linguists have to look for hierarchical (non-linear) context information. 

Therefore, \citep{2017opennmt} theoretically incorporation of the gloss data should improve the translation systems. Specifically, I propose the following hypothesis:
\begin{exe}  
\ex \textbf{Gloss-helps hypothesis: the translation systems trained with the gloss data incorporated should outperform the systems trained with only Gaelic and English sentences pairs (i.e. without gloss data).}

The hypothesis can have two versions, strong and weak:
	\begin{xlist}
	\ex \label{strong_hy} Strong version: Gloss may replace the source natural language totally, and the system outperforms the system trained with source natural language to target language sentence pairs (i.e. the baseline systems).   
	\ex \label{weak_hy} Weak version: Gloss only increases the performance of the baseline systems, but cannot replace the source language. 
	\end{xlist}
\end{exe}
\subsection{literature}

\bibliographystyle{te}
\bibliography{ref}

\end{document}
