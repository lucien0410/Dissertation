\chapter{What are glosses? Why are them golden representations of meanings?}
\label{chap:gloss}

\section{Key Points of The Chapter}
	\begin{enumerate}
	\item Target Audience: CS people
	\item main points:
		\begin{enumerate}
    	\item summary of the Leipzig Glossing Rules \citep{bickel2008leipzig}.  
		\item glossing is the initial processing of the data guided by some specific syntax theory. 
    	\item Glosses contain morphology information (with examples); glosses disambiguate homographs (with many examples); gloss also provide some parsing information because some glosses are determined by structural/constituency context (with examples).       
		\end{enumerate}
 
	\end{enumerate}

\section{Introduction: What are Glosses}

Interlinear Glossed Text (IGT) is widely used in linguistic studies. (\ref{gloss_eg}) is an example of Scottish Gaelic IGT.
\begin{exe}  
\ex\label{gloss_eg} \gll    Tha a athair nas sine na a mh\`athair.\\  
            be.pres 3sm.poss father comp old.cmpr comp 3sm.poss mother
\\  
    \glt    `His father is older than his mother.'  
\end{exe}

(summarize and exemplify the Leipzig Glossing Rules)


\section{The Golden Properties of Glosses}

A system of meaning representations is decomposed of three components: a) meanings, b) representations, and c) a mapping between meanings and representations. The most ideal meaning representation system should be built with one-meaning-to-one-representation mappings; in other words, a meaning is mapped to one and only one representation. Natural languages fail to do so, given that synonyms and ambiguous words/phrases are ubiquitous in natural languages. On the other hand, gloss provides a mapping that is close to this ideal one-to-one mapping. Thus gloss should a better representation in term of representing meanings. 

Theoretically, the claim that gloss representation is closer to the ideal one-to-one mapping than natural language representation is can be tested empirically. IF there were a set of special golden meta-linguistic semantic representations, which has the following property: each concept is mapped to one and one representation and each representation is mapped to one and one concept, then it is expected that each gold representation will map to more natural language words than gloss items, and each natural language word will map to more gold representations than gloss items. However, in practice, this is an impossible experiment to conduct, because there are no such golden representation\footnote{It would solve the puzzle of semantics if one should be able to build the set of special golden meta-linguistic semantic representations, and the mappings between the golden representations to natural languages.}. However, in spite of the impossibility of conducting statistical experiments, we may still use some examples to show the intuition that glosses are better representations than natural languages. The following sections describes how glosses cluster words with different forms but with the same meaning, and how glosses represent words with same form but with different meanings with different representations. 

\subsection{Glosses Cluster Different Words with the Same Meanings (Synonyms) Into a Single representation}
Gloss collapses words with different forms with the same meanings into a single gloss. In natural languages, the morphology of a word (i.e. the form of a word) may be sensitive to the phonological environments and changing into different forms. Consider the following the indefinite article in the English examples: 

\begin{exe}  
\ex \gll John ate \textbf{an} apple.\\
	John eat.past	\textbf{ART} apple\\
\ex \gll John ate \textbf{a} banana.\\
	John eat.past   \textbf{ART} banana\\
\end{exe}

In the above example, \textit{an} and \textit{a} have the identical meaning\footnote{Semantically, \textit{an} and \textit{a} are existential quantifiers, which declare that a member of a set exists in the world. In formal semantics, \textit{an} and \textit{a} may be defined as follows: $\exists\lambda P[P(x)]$. In the current example, \textit{apple} and \textit{banana} will instantiate $P$ in the formula, and the meanings will be `an apple exists' and `a banana exists'. \citet{kratzer1998semantics} would be a nice introduction for interested readers to see how linguists, specifically semanticians, define, decompose, and compose meanings of languages formally.}. 
In English, the same concept is realized as two representations, \textit{a} or \textit{an}, while in the gloss representation the one concept is neatly represented as \textit{ART}. 

Critically, synonyms like the English \textit{a} and \textit{an} commonly occur in many other natural languages if not in all languages. The definite article in the language of interest, Scottish Gaelic, is another example to show the noisiness of natural language representations. Consider the definite article in the following Gaelic examples. 

\begin{exe}  
\ex 
\gll tha mi a' sireadh \textbf{an} leabhair bhig ghuirm\\
be-PRES-IND 1S PROG searching-VN \textbf{ART} book-G small-G blue-G\\
\glt `I am looking for the small blue book' \citep[p. 29]{lamb2001scottish}

\ex 
\gll \textbf{am} fear m\`or\\
\textbf{ART} man big\\
\glt `a big man' \citep[p. 31]{lamb2001scottish}

\ex
\gll thuit \textbf{a'} chlach air cas mo mhn\`a\\
fall-PAST \textbf{ART} stone on foot 1S-POSS wife-G\\
\glt`the stone fell on my wife's foot' \citep[p. 30]{lamb2001scottish} 	

\ex
\gll doras \textbf{na} sgoile(adh) \\
door-N \textbf{ART} school-G \\
\glt `the door of the school' \citep[p. 29]{lamb2001scottish} 	

\ex 
\gll a chuir air d\`oigh \textbf{nan} \`airidhean a-muigh a rubh' Eubhal agus an oidhche seo \\
to put-INF on order \textbf{ART} sheilings out-LOC to point Eaval and ART night this \\
\glt `the girls big house' \citep[p. 100]{lamb2001scottish} 

\ex
\gll f\`eis \textbf{nam} b\`ard\\
festival \textbf{ART} poet.PL.GEN\\
\glt `festival of the poets' \citep[p. 107]{lamb2001scottish}

\end{exe}

The definite article in Scottish Gaelic may be realized as the following forms: as \textit{an}, \textit{am}, \textit{a'}, \textit{na}, \textit{nan} or \textit{nam}. The alternation is determined by the case, gender and number of noun phrase that it modifies, and additionally the phonological property of the word following it also changes the form of the definite article \citep{lamb2001scottish}. All these different realizations refer to the same concept, the definite article. Again, the gloss notation nicely clusters them together as \textit{ART}. 

In Mandarin Chinese, similar patterns are found. Consider classifiers in the following examples:

\begin{exe}
\ex \label{chinese_cl_eg}
\gll Yani mai-le \{\textbf{pi}/\textbf{*tou}\} ma , Lulu mai-le \{\textbf{*pi}/\textbf{tou}\} zhu.\\ 
Yani buy-PRF CL/CL horse , Lulu buy-PRF CL/CL\\
\glt `Yani bought a horse and Lulu bought a pig.' \citep[p. 136]{zhang2013classifier}
\end{exe}

In \citet{zhang2013classifier}, the classifier like \textit{pi} and \textit{tou} is a type of \textit{indivual classifier} which co-occurs with countable nouns, like \textit{ma}, `horse', and \textit{zhu}, `pig', and this type of classifier is the head of \textit{UNIT Phrase}. 
\textit{Pi} and \textit{tou} actually have the same semantics and the syntactic function; however, they are realized in different forms, specifically the form of which has to agree with the noun following it (i.e. \textit{pi} goes with \textit{ma}; \textit{tou} goes with \textit{zhu}). Here the gloss, \textit{CL}, unifies the two forms of the same meaning.    

In short, gloss collapses synonyms in natural languages. Learning the general distribution of the article and all its different forms is a challenge for the MT system, but the glossing information should make this easier.

\subsection{Glosses Distinguish Homographs' Different Meanings}
In natural languages, there are cases when a single form denotes to distinct concepts. Words with this properties are termed as homographs. Consider the word \textit{for} in following English examples:

\begin{exe}
\ex \label{for_eng}
	\begin{xlist}
	\ex \label{for_c}I intended \textbf{for} Jenny to be present.
	\ex \label{for_p}\textbf{For} Jenny, I intended to be present. \citep[p.306-307]{adger2003core}
	\end{xlist}
\end{exe}

\textit{For} in (\ref{for_c}) and (\ref{for_p}) has the same form but different meanings. Specifically, \textit{for} in (\ref{for_c}) is a complementizer with its part of speech being \textit{C}, and it heads the non-finite clause \textit{Jenn to be present}; on the other hand \textit{for} (\ref{for_p}) is a preposition, which takes the Determiner Phrase, \textit{Jenny}, as its benefactive argument.   

The Scottish Gaelic word \textit{a'} in the following examples also has different meanings.  

\begin{exe}  
\ex \label{a_prog}
\gll tha mi \textbf{a'} sireadh an leabhair bhig ghuirm.\\
be-PRES-IND 1S \textbf{PROG} searching-VN ART book-G small-G blue-G\\
\glt `I am looking for the small blue book.' \citep[p. 29]{lamb2001scottish}

\ex \label{a_det}
\gll thuit \textbf{a'} chlach air cas mo mhn\`a.\\
fall-PAST \textbf{ART} stone on foot 1S-POSS wife-G\\
\glt`the stone fell on my wife's foot.' \citep[p. 30]{lamb2001scottish} 	
\end{exe}

Critically \textit{a'} in (\ref{a_prog}) is a progressive aspect marker while in (\ref{a_det}) the some form denotes to definite article. Again, the semantic difference is preserved in the gloss representations but not in natural language words.  
The gloss data also provides hierarchical (non-linear) syntactic parsing information. 

\subsection{Glosses are Sensitive to Hierarchical Structure of Natural Language Sentences}

It is well-argued in linguistics that the syntax and semantics of natural languages are sensitive to hierarchical structures instead of linear orders of words, and essentially it is the sensitivity of hierarchical structures that distinguishes human natural languages from other animal communications \citep{berwick2015only}.     

Semantics is determined by hierarchical structures instead of linear orders. \citet[p. 117]{berwick2015only} use the following simple example to demonstrate this property of natural languages:

\begin{exe}
\ex Instinctively birds that fly swim. 
\end{exe}

In the example above, \textit{instinctively} is linearly close to \textit{fly} than \textit{swim}; however, it unambiguously modifies \textit{swim} instead of \textit{fly}. The reason for this is the hierarchical structures \citep[p. 117]{berwick2015only}:

\begin{exe}
\ex \label{tree}
\Tree [Instinctively [ [  birds   [  that   fly  ]  ]  swim  ] ]
\end{exe}

In (\ref{tree}) it is shown that \textit{fly} is more embedded than \textit{swim}, and thus it is hierarchically further away from \textit{instinctively}. So, \textit{instinctively} can only modify \textit{swim} instead of \textit{fly}.

Syntax is also all about hierarchical structures. Consider the following declarative sentence:

\begin{exe}
\ex Birds that can\textsubscript{1} fly can\textsubscript{2} swim. 
\end{exe}


If externalization and communication are the primary nature of language, we should expect that some natural human languages should have manipulated the linear order of words in a sentence to express different meanings, because linear order is the externalization of language. However, this is not the case. No language exploits linear order, and instead universally language uses hierarchical structures, which is more difficult to be externalized than linear order. Chomsky uses the following examples to illustrate his points.

\begin{exe}
\ex
	\begin{xlist}
	\ex\label{c1} Smart eagles can swim.
	\ex\label{c2} *Eagles smart can swim?
	\ex\label{c3} Can smart eagles swim?
	\end{xlist}
\end{exe}

(\ref{c1}) is a declarative sentence. (\ref{c2}) tries to swap the first two words to derive an interrogative sentence, and fails. (\ref{c3}) is the real interrogative sentence in natural languages. If externalization and communication are the primary nature of language, it is expected (\ref{c2}) to be a common pattern in languages, because (\ref{c2}) exploits linear order, which is the externalization of language and the linear order serves the communication purpose just fine given that humans' cognition system is able to tell the difference between (\ref{c1}) and (\ref{c2}). However, the pattern in (\ref{c2}) is not found in any human language; instead, (\ref{c3}) is used, which uses hierarchical structure relations. In term of externalization, (\ref{c2}) is more economical than (\ref{c3}) given that (\ref{c2}) only moves two positions of words while (\ref{c3}) moves three positions of words. Again, if externalization and communication are the primary nature of language, language should have evolved the pattern of (\ref{c2}). Based on the fact that no language exploits linear order, Chomsky argues that language externalization and communication in particular are accessory.   

Glosses are also sensitive to the internal hierarchical structures or constituency of sentences. Consider the following example, modified from \ref{for_eng}:

\begin{exe}
\ex
	\begin{xlist}
	\ex \label{for_c}I intended \textbf{for} [the little girl who wants to eat some ice scream] to be present.
	\ex \label{for_p}\textbf{For} [the little girl who wants to eat some ice scream], I intended to be present. 
	\end{xlist}
\end{exe}



\section{Conclusion: What is a Gloss Line?} 
A gloss line is an artificial sentence using the purified `gloss words', a meaning representation with which one meaning is mapped to one and only one representation. 

% section conclusion (end)
