\chapter{What are glosses? Why are them golden representations of meanings?}
\label{chap:gloss}

\section{Key Points of The Chapter}
	\begin{enumerate}
	\item Target Audience: CS people
	\item main points:
		\begin{enumerate}
    	\item summary of the Leipzig Glossing Rules \citep{bickel2008leipzig}.  
		\item glossing is the initial processing of the data guided by some specific syntax theory. 
    	\item Glosses contain morphology information (with examples); glosses disambiguate homographs (with many examples); gloss also provide some parsing information because some glosses are determined by structural/constituency context (with examples).       
		\end{enumerate}
 
	\end{enumerate}

\section{Introduction: What are Glosses}

Interlinear Glossed Text (IGT) is widely used in linguistic studies. (\ref{gloss_eg})  is an example of Scottish Gaelic IGT.
\begin{exe}  
\ex\label{gloss_eg} \gll    Tha a athair nas sine na a mh\`athair.\\  
            be.pres 3sm.poss father comp old.cmpr comp 3sm.poss mother
\\  
    \glt    `His father is older than his mother.'  
\end{exe}

(summarize and exemplify the Leipzig Glossing Rules)


\section{The Golden Properties of Glosses}

A system of meaning representations is decomposed of three components: a) meanings, b) representations, and c) a mapping between meanings and representations. The most ideal meaning representation system should be built with one-meaning-to-one-representation mappings; in other words, a meaning is mapped to one and only one representation. Natural languages fail to do so, given that synonyms and ambiguous words/phrases are ubiquitous in natural languages. On the other hand, gloss provides a mapping that is close to this ideal one-to-one mapping. Thus gloss should a better representation in term of representing meanings. 

Theoretically, the claim that gloss representation is closer to the ideal one-to-one mapping than natural language representation is can be tested empirically. IF there were a set of special golden meta-linguistic semantic representations, which has the following property: each concept is mapped to one and one representation and each representation is mapped to one and one concept, then it is expected that each gold representation will map to more natural language words than gloss items, and each natural language word will map to more gold representations than gloss items. However, in practice, this is an impossible experiment to conduct, because there are no such golden representation\footnote{It would solve the puzzle of semantics if one should be able to build the set of special golden meta-linguistic semantic representations, and the mappings between the golden representations to natural languages.}. However, in spite of the impossibility of conducting statistical experiments, we may still use some examples to show the intuition that glosses are better representations than natural languages. The following sections describes how glosses cluster words with different forms but with the same meaning, and how glosses disambiguate words with same form but with different meanings. 

\subsection{Glosses Cluster Different Words with the Same Meanings (Synonyms) Into a Single representation}
Gloss collapses words with different forms with the same meanings into a single gloss. In natural languages, the morphology of a word (i.e. the form of a word) may be sensitive to the phonological environments and changing into different forms. Consider the following the indefinite article in the English examples: 

\begin{exe}  
\ex \gll John ate \textbf{an} apple.\\
	John eat.past	\textbf{ART} apple\\
\ex \gll John ate \textbf{a} banana.\\
	John eat.past   \textbf{ART} banana\\
\end{exe}

In the above example, \textit{an} and \textit{a} have the identical meaning\footnote{Semantically, \textit{an} and \textit{a} are existential quantifiers, which declare that a member of a set exists in the world. In formal semantics, \textit{an} and \textit{a} may be defined as follows: $\exists\lambda P[P(x)]$. In the current example, \textit{apple} and \textit{banana} will instantiate $P$ in the formula, and the meanings will be `an apple exists' and `a banana exists'. \citet{kratzer1998semantics} would be a nice introduction for interested readers to see how linguists, specifically semanticians, define, decompose, and compose meanings of languages formally.}. 
In English, the same concept is realized as two representations, \textit{a} or \textit{an}, while in the gloss representation the one concept is neatly represented as \textit{ART}. 

Critically, synonyms like the English \textit{a} and \textit{an} commonly occur in many other natural languages in not in all languages. The definite article in the language of interest, Scottish Gaelic, is another example to show the noisiness of natural language representations. Consider the definite article in the following Gaelic examples from \cite{lamb2001scottish}. 

\begin{exe}  
\ex 
\citet[p. 29]{lamb2001scottish}
\gll tha mi a' sireadh \textbf{an} leabhair bhig ghuirm\\
be-PRES-IND 1S PROG searching-VN \textbf{ART} book-G small-G blue-G\\
\glt `I am looking for the small blue book'

\ex 
\citet[p. 31]{lamb2001scottish}
\gll \textbf{am} fear m\`or\\
\textbf{ART} man big\\
\glt `a big man'

\ex
\citet[p. 30]{lamb2001scottish} 
\gll thuit \textbf{a'} chlach air cas mo mhn\`a\\
fall-PAST \textbf{ART} stone on foot 1S-POSS wife-G\\
\glt`the stone fell on my wife's foot' 	

\ex
\citet[p. 29]{lamb2001scottish} 
\gll doras \textbf{na} sgoile(adh) \\
door-N \textbf{ART} school-G \\
\glt `the door of the school'		

\ex 

\gll a chuir air d\`oigh \textbf{nan} \`airidhean a-muigh a rubh' Eubhal agus an oidhche seo \\
to put-INF on order \textbf{ART} sheilings out-LOC to point Eaval and ART night this \\
\glt `the girls big house'




\end{exe}

The definite article in Gaelic may be realized as the following forms: as \textit{an}, \textit{am}, \textit{a'}, \textit{nam}, \textit{nan} or \textit{na}. The alternation is determined by the case, gender and number of noun phrase that it modifies, and additionally the phonological property of the word following it also changes the form of the definite article \citep{lamb2001scottish}. 

\subsection{Glosses disambiguates ambiguous words}
Critically glosses provide `redundant' information. Here `redundant' means that  

\subsection{Glosses are sensitive to hierarchical structure of natural language sentences}
Critically glosses provide `redundant' information. Here `redundant' means that  

\section{What is a Gloss Line}
A gloss line is an artificial sentence using the purified `gloss words'. 

\section{Conclusion} 

% section conclusion (end)
